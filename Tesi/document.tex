\documentclass[10pt,a4paper]{article}
\usepackage{graphicx} %LaTeX package to import graphics
\graphicspath{image/} %configuring the graphicx package,le immagini si trovano in image/
\usepackage{float}
\usepackage[utf8]{inputenc}
\usepackage[T1]{fontenc}
\title{Ambliopia}
\author{Luca Frangiamore}

\date{2023}

\begin{document}
	
	\maketitle
	\tableofcontents{\tiny }
	
	
	\begin{abstract}
		Trattazione condizione dell'ambliopia anche detta "occhio pigro" mediante l'utilizzo di un applicativo mobile.
	\end{abstract}
	 \newpage
	\section {Ambliopia}
	L'ambliopia, anche conosciuta come "occhio pigro" è una patologia che si manifesta nei prima anni dello sviluppo, età compresa tra 0-6 anni, il quale colpisce circa il 4\% della popolazione mondiale.
	La patologia riguarda la non corretta stimolazione dell'apparato visivo, in questo caso il cervello non riesce ad ricreare l'immagine tridimensionale e quindi preferisce trascurare un occhio, l'occhio malato.
	Nella maggior parte dei casi, l'occhio è anatomicamente perfetto, \textbf{ambliopia funzionale}, in casi peggiori possiamo riscontrare \textbf{ambliopia organica} in cui vi è una deviazione delle vie ottiche.
	La seguente patologia si può manifestare con più probabilità in maniera asimmetrica, \textbf{monolaterale}, il quale colpisce un solo occhio o più raramente possiamo trovare la forma \textbf{bilaterale}, il quale colpisce entrambi gli occhi.
	\subsection{Cause}
	In generale come detto prima, l'occhio pigro riguarda il progressivo trascuramento dei segnali di uno dei due occhi.
	Questo processo di sviluppo è causato da un non corretto sviluppo delle vie nervose degli occhi, le quali vengono stimolate in modo non bilanciate, ciò è dovuto magari dalla presenza di una condizione oculare presente in uno dei due occhio(nel caso di \textbf{monolaterale }.
	Alcune condizioni che possono insorgere sono:
	\begin{itemize}
		\item Astigmatismo:Visione è poco nitida e distorta in qualsivoglia direzione.
		\item Strabismo:Deviazione degli assi visivi, impedisce il corretto coordinamento degli occhi. 
		\item Cataratta:Opacizzazione parziale o totale del cristallino,  causa offuscamento e difficoltà nel mettere a fuoco le immagini.
		\item Ptosi palpebrale:Una o entrambi le palpebre sono superiori sono abbassate più del normale.
	\end{itemize}
	Come detto prima, se il cervello non riesce a combinare le immagini proveniente dai due occhi, esso può decidere di trascurare un dei due segnali, prediligendo l'occhio ottimale, sviluppando quindi l'ambliopia.
	\subsection{Sintomi}
	Vi sono alcuni casi, in cui il paziente probabilmente giovane, non si accorge della patologia, questo può ritardare o eliminare i trattamenti attui a ridurre o eliminare il problema.
	Tra i problemi più comuni possiamo citare:	   
	\begin{itemize}
		\item Difficoltà dei visione in un occhio.
		\item Movimenti involontari dell'occhio.
		\item Sensibilità al movimento compromessa.
		\item Scarsa percezione della profondità in quanto il cervello privilegia un occhio a causa della ridotta acuità visiva dell'altro.
	\end{itemize}    
	\subsection{Trattamenti classici }
	Subito dopo aver riconosciuto il disturbo bisogna procedere con la corretto terapia. Per prima cosa si corregge il difetto che ha portato l'inibizione dell'occhio pigro, successivamente si procede con la terapia occlusiva permette di stimolare l'occhio pigro in modo da costringerlo a lavorare.
	\begin{itemize}
		\item Patching, La terapia consiste nel coprire l'occhio dominante, da applicare per un perioda di tempo variabile.
		Il trattamento è efficacie ma il recupero della vista impiega diversi mesi.       	   
		\begin{figure}[h]
			\centering
			\includegraphics[width=0.7\linewidth]{image/patching}
			\caption{Trattamento con Patching}
			\label{fig:patching}
		\end{figure}
	
		\item Penalizzazione ottica, consiste nel indossare dei supporti fisici, con delle lenti con diversi gradi di opacizzazione.
		
		\begin{figure}[h]
			\centering
			\includegraphics[width=0.7\linewidth]{image/penalizzazione ottica}
			\caption{Trattamento con lente di bangerter}
			\label{fig:penalizzazione-ottica}
		\end{figure}
	\newpage
		\item Collirio, permette di offuscare la vista dell'occhio dominante, in modo da poter stimolare il l'occhio più debole
		
		\item Luminopia, è un software approvato da  \textbf{Food and Drug Administration(FDA)}, che permette ai pazienti di poter usufruire di 700 ore di serie o film, adattati tramite AI, questo permette di rendere il trattamento dell'ambliopia piacevole e quindi sopportabile nel lungo periodo.
		I pazienti usufruiscono di questi contenuti mediante il vr, in modo da poter visionare i contenuti scelti.
		\begin{figure}[h]
			\centering
			\includegraphics[width=0.7\linewidth]{image/luminopia}
			\caption{Trattamento innovativo con Luminopia}
			\label{fig:luminopia}
		\end{figure}
	\end{itemize}

	
   
	Questi trattamenti devono essere usati in un contesto di prevenzione.
	Maggiore è l'età della diagnosi, minori sono le possibilità di recupero dall'ambliopia.\\
	\thanks{Cristina D'avena,Matteo Verzeroli }
\end{document}